% !TeX program = lualatex

\documentclass[
ngerman,
subtask=ruled %or plain
]{tudaexercise}

\usepackage[english, main=ngerman]{babel}
\usepackage[autostyle]{csquotes}

\usepackage{amsmath,amssymb}

\usepackage{float}

\usepackage{biblatex}
%\bibliography{DEMO-TUDaBibliography}

\newcommand{\dd}{\,\mathrm{d}}
\newcommand{\e}{\mathrm{e}}


\ConfigureHeadline{
	headline={HSI Übung - Niklas Beck und Paul Hollmann}
}

\begin{document}
	
	\title[Übung Hochleistungssimulationen]{Hochleistungssimulationen}
	\subtitle{\"Ubung 1}
	\author{Niklas Beck (2582775), Paul Hollmann (2465070)}
	\term{Wintersemester 2023/24}
	%\sheetnumber{1}
	\date{Abgabe 26. November 2023}
	\maketitle
	
	\hrule
	{\Large \textbf{Matrix-Vektor-Multiplikation}}
	\hrule
	
	Erklärung als eine Versuchsbeschreibung, Versuchsvorhaben und Arbeitsschritte
	
	\begin{task}{Sequentielle Berechnung des Matrix-Vektor-Produkts}
		Das Matrix-Vektor-Produkt ist eine wohl bekannte mathematische Operation, welche auch für große Matrizen gebraucht wird.
		Eine kleine Eigenheit der hier verlangten Funktion ist die Eingabe nur eines Wertes $m$, welcher zur Initialisierung einer quadratischen Matrix und eines Vektors mit Zufallswerten genutzt wird.
		Ohne Vorgabe eines Datentyps haben wir uns für den Typ float entschieden, da er einen größeren Eingabemenge ermöglicht, aber nicht allzu groß ist. Wäre die Präzision relevant, hätten wir uns eher für den double entschieden.
	\end{task}
	
	
	\begin{task}{Parallele Berechnung des Matrix-Vektor-Produkts}
		Noch bevor Bearbeitung der nächsten Teilaufgabe haben wir uns die Dokumentationen von OpenCL und JOCL herausgesucht und anhand der Beispiele aus dem moodle-Kurs und von JOCL ein Verständnis zur Einbindung von OpenCL-Code erlangt.
		
		Unser Kernel ist in einer Extradatei ausgelagert, da es eine schönere Formatierung ermöglicht.
		Dieser muss per BEFEHL in den JOCL Code eingefügt werden.
	\end{task}
	
	
	\begin{task}{Ausführung auf dem Lichtenberg-Hochleistungsrechner} 
		Zur Ausführung auf dem Cluster wird ein Job Script gebraucht.
		
	\end{task}

	\begin{task}{Messungen und Analyse} 
		In dieser Teilaufgabe ist verlangt für einige Problemgrößen die Ausführungszeit zu messen.
		
		\begin{tabular}{|c|c|c|}
			\hline
			Problemgröße $m$ & sequenzielle Abarbeitung & OpenCL - Parallelisierung \\
			\hline
			10 &  &  \\
			\hline
			1000 &  &  \\
			\hline
			2000 &  &  \\
			\hline
			4000 &  &  \\
			\hline
			8000 &  &  \\
			\hline
			15000 &  &  \\
			\hline
		\end{tabular}
	\end{task}

	\begin{task} {Variation der Anzahl der work-groups}
		Variieren Sie manuell die Größe und Anzahl der verwendeten work-groups (siehe oben local\_work\_size) und ermitteln
		Sie jeweils die Dauer der parallelen Abarbeitung bei einer konstanten Problemgröße n (m = 10000). Die Ausführung
		des parallelisierten Programms soll auf dem Lichtenberg-Hochleistungsrechner erfolgen. Dokumentieren Sie die zur
		Ausführung verwendete Hardware und relevante OpenCL Informationen (CL\_DEVICE\_MAX\_WORK\_GROUP\_SIZE etc.).
		Was ist zu beobachten und wie erklärt sich das Ergebnis?
	\end{task}

	
\end{document}